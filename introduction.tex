
The \h\ ion is a hydrogen atom with an extra electron.  This extra electron
has a relatively low ionization potential of $\chi = 0.75$ eV and
provides a significant contribution to the opacity of the sun's
photosphere (\citealt{ryden2010foundations}).  It is relevant in all
stars cooler than F0 (\citealt{carroll2007introduction}). \h\
abundance (and therefore its contribution to the opacity) is sensitive 
to temperature and to the abundance of low-\chii\ metals
(\citealt{hansen1994stellar}).  The temperature dependence of \h\
abundance can be characterized qualitatively as follows: the gas needs
to be of high enough temperature to allow free electrons (ionized from
low-\chii\ metals as well as some hydrogen) to exist.  The gas,
however, also needs to be of low enough temperature that the
lightly-bound second electrons are not quickly stripped from the
hydrogen atoms either through collisional ionization or a large flux
of photons.  Thus, if the temperature is too low \h\ ions will not be
produced and if the temperature is too high \h\ ions won't survive
long enough to make significant contributions to the opacity.


Note to Joshua:  \cite{hansen1994stellar} figure 4.4 provides a figure
of \h\ opacities, I think the temperature dependence.
