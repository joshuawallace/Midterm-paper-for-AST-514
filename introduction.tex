
The \h\ ion has long been recognized as an important source of
continuum opacity in the sun at visible
wavelengths. The existence of \h\ was proven by \cite{bethe1929}. Its
astrophysical importance as a significant contribution to solar
opacities was recognized early on, \citet{Wildt1939a,Wildt1939b} being among the first to
recognize this.  The \h\ ion is, quite simply, a hydrogen atom with an extra electron.  This extra electron
has a relatively low ionization potential of $\chi = 0.754$ eV
(\citealt{carroll2007introduction}, which corresponds to a photon
wavelength of $\sim 16500$\AA.  It is, in part, due to this low
ionization potential that \h\ provides such  a significant contribution to the opacity of the sun's
photosphere in the visible and near-infrared.  \h\ opacities are relevant in all
stars cooler than F0 (\citealt{carroll2007introduction}). Bound-free
absorption on \h\ is signficant in the wavelength range
$\sim$3000-16500\AA, while free-free absorption on \h\ is significant
for wavelengths longer than $\sim$16500\AA.  

 \h\ abundance (and therefore its contribution to the opacity) is sensitive 
to temperature and the presence of free electrons, contributed mainly
by metals with low ionization potential $\chi$.
(\citealt{hansen1994stellar}), primarily the alkali and alkaline earth
metals.  

%The temperature dependence of \h\
%abundance can be characterized qualitatively as follows: the gas needs
%to be of high enough temperature to allow free electrons (ionized from
%low-\chii\ metals as well as some hydrogen) to exist.  The gas,
%however, also needs to be of low enough temperature that the
%lightly-bound second electrons are not quickly stripped from the
%hydrogen atoms either through collisional ionization or a large flux
%of photons.  Thus, if the temperature is too low \h\ ions will not be
%produced and if the temperature is too high \h\ ions won't survive
%long enough to make significant contributions to the opacity.

