%\documentclass[preprint]{aastex}
\documentclass{emulateapj}
%\documentclass{article}
\usepackage{graphicx}
\usepackage{natbib}
%\bibliographystyle{apj}
\usepackage{amssymb}
\usepackage{float}


\def\h{H$^-$}
\def\chii{$\chi$}
\def\beq{\begin{equation}}
\def\eeq{\end{equation}}

\begin{document}

\title{\h\ Opacity in the Solar Atmosphere}

\author{Joshua Wallace}

\affil{AST 514, Department of Astrophysical Sciences, Princeton
  University, Princeton, NJ 08544}

\begin{abstract}
A review of H$^-$ in stellar atmospheres. Make it long  Make it long  Make it long  Make it long  Make it long  Make it long  Make it long  Make it long  Make it long  Make it long  Make it long  Make it long  Make it long  Make it lon Make it long g  Make it long  Make it long  Make it long  Make it long  Make i Make it long  Make it long  Make it long  Make it long  Make it long  Make it long  Make it long  Make it long  Make it long  Make it long  Make it long t long 
\end{abstract}

\keywords{Sun: atmosphere --- stars: atmospheres}

\section{Introduction}

The \h\ ion is a hydrogen atom with an extra electron.  This extra electron
has a relatively low ionization potential of $\chi = 0.75$ eV and
provides a significant contribution to the opacity of the sun's
photosphere (\citealt{ryden2010foundations}).  It is relevant in all
stars cooler than F0 (\citealt{carroll2007introduction}). \h\
abundance (and therefore its contribution to the opacity) is sensitive 
to temperature and to the abundance of low-\chii\ metals
(\citealt{hansen1994stellar}).  The temperature dependence of \h\
abundance can be characterized qualitatively as follows: the gas needs
to be of high enough temperature to allow free electrons (ionized from
low-\chii\ metals as well as some hydrogen) to exist.  The gas,
however, also needs to be of low enough temperature that the
lightly-bound second electrons are not quickly stripped from the
hydrogen atoms either through collisional ionization or a large flux
of photons.  Thus, if the temperature is too low \h\ ions will not be
produced and if the temperature is too high \h\ ions won't survive
long enough to make significant contributions to the opacity.

As far as a qualitative picture for opacity dependence on metallicity,
well, that's where figure 4.4 comes in.

Note to Joshua:  \cite{hansen1994stellar} figure 4.4 provides a figure
of \h\ opacities, I think the temperature dependence.


\section{Brief History of the Development of the Understanding of \h's Importance in the Sun}
The existence of \h\ was proven by \cite{bethe1929}.  %He used a three-parameter wavefunction of the form $(1+\alpha
Its astrophysical importance was recognized by \citet{Wildt1939a,Wildt1939b}.


\section{The Nature of the \h\ Ion}
The non-relativistic, Coulomb-interaction only Hamiltonian for \h\ is given by 
\begin{equation}
\label{eq:H}
H = \frac{p_1^2}{2m_e} + \frac{p_2^2}{2m_e} - \frac{1}{4\pi\epsilon_0}\frac{e^2}{r_1} -  \frac{1}{4\pi\epsilon_0}\frac{e^2}{r_2} +  \frac{1}{4\pi\epsilon_0}\frac{e^2}{r_{12}}
\eeq
where $m_e$ is the mass of the electron, $p$ is the classical kinetic energy of the electron specified by the subscript, $e$ is the charge of the electron, $\epsilon_0$ is the vacuum permittivity, $r_1$ is the distance from the first electron to the nucleus, $r_2$ is the corresponding value for the second electron, and $r_{12}$ is the distance between the two electrons.  In this Hamiltonian the nucleus is assumed to be fixed and thus does not possess its own kinetic energy.  This Hamiltonian possesses a term for each electron's kinetic energy, each electron's potential energy from its interaction with the nucleus (a single proton), and (specifically the last term in equation~\ref{eq:H}) a term for the repulsive Coulomb interaction between the two electrons.

Many approaches have been taken to calculate wavefunctions for \h.  Many-paramter Hylleraas functions seem to be the most commonly used.  I will focus on a simple two-parameter trial wavefunction used by \cite{chandra1944} because of the interpretations to which the simple wavefunction lends itself.  The wavefunction is
\beq
\phi = \exp{-\alpha r_1 - \beta r_2} + \exp{\alpha r_2 - \beta r_1}.
\eeq
Chandrasekhar showed that the minimum energy of this wavefunction (at $\alpha = 1.03925$ and $\beta = 0.28309$) is sufficient to provide binding for \h.  The wavefunction gives the electrons a radial hierarchy, with one electron in close to the nucleus and the other far away from the nucleus (which gives it its correspondingly smaller binding energy). % Additionally, the innermost electron

It has been rigorously shown that \h\ has no bound excited states, i.e., the only bound state is the ground state (\citealt{hill1977}).  As such, \h\ does not have a spectrum as it is normally referred to; there are no electronic transitions between bound states that absorb and emit photons of specific wavelengths.  Its main contribution to the opacity of the solar atmosphere is through continuum absorption of photons with energy greater than the binding energy of its least bound electron.  

The effect of screening due to free electrons on the binding energy of the \h\ ion has been looked at.  The partially ionized nature of the solar atmosphere makes this a relevant study.  \cite{phelpsbajaj1983} use a variational approach to show that the effect screening due to free electrons has on the binding energy of the \h\ ion is within a factor of 2 (as they quantify it) of the effect of screening on the binding energy of a single-electron H atom, despite the electron of the neutral H being $\sim 18$ times more bound than the least bound electron of \h.  To be specific, \cite{phelpsbajaj1983} 
%define a screening parameter $\delta$, where
%\beq
%\delta^2 = \sum_i \frac{4\pi n e^2}{k_B T}\left[ \frac{F_{-1/2}(\nu_i)}{F_{1/2}(\nu_i)}
%\eeq
assume a Dingle-Mansfield screening, which modifies equation~\ref{eq:H} to
\beq
H = \frac{p_1^2 + p_2^2}{2m_e} - \frac{1}{4\pi\epsilon_0}\frac{e^2}{r_1}e^{-\delta r_1} -  \frac{1}{4\pi\epsilon_0}\frac{e^2}{r_2}e^{-\delta r_2} + \\
 \frac{1}{4\pi\epsilon_0}\frac{e^2}{r_{12}} e^{-\delta r_{12}}
\eeq
where $\delta$ is a screening parameter defined in the text.  For my
purpose of comparison the exact definition of the screening parameter
is not necessary to include.  Their results are shown in
figure~\ref{fig:screening}.  They calculate that for  \h\ the binding energy goes to zero for $a_0 \delta = 0.87$  (where $a_0$ is the Bohr radius) while it is shown elsewhere that for H binding energy goes to zero for $a_0 \delta = 1.2$ (\citealt{rogersetal1970}).  The two values of $a_0 \delta$ are not very different which, as mentioned previous, seems surprising due to the factor $~18$ discrepancy between the binding energies of H and \h.  \cite{phelpsbajaj1983} attribute this to the assumption that the interactions between the proton and the electrons and the interaction between the electrons themselves are modified in the presence of free electrons and so the ground state energy is itself modified as a function of $\delta$; so the binding energy is modified both by the presence of free electrons and by the change in the ground state energy in the presence of free electrons.  The interplay of these two effects is what allows a perhaps intially surprisingly large value of $a_0 \delta$ for the binding energy to go to zero in \h.
\begin{figure}
\includegraphics[width=76mm]{figs/screeningparameter.png}%normal width
                                %is 80mm
\caption{\label{fig:screening}Binding energy of the least bound
electron in the \h\ ion as a function of screening parameter
$\delta$.  Binding energy is given in units of the Rydberg energy and
multiplied by 100 (\citealt{phelpsbajaj1983}).}
\end{figure}


\section{Abundance of \h\ Ion in Stellar Atmosphere}
The \h\ abundance depends on several factors: the abundance of H, the
abundance of free electrons (ionized from other species present in the
atmosphere, including H), and the factors that lead to the conversion
of \h\ to H: photoionization and collisions with other particles.  All
of these factors have some sort of dependence on temperature.

%The solar atmosphere is made up about 70\% by mass of H.  As will be
%shown in what follows, the fraction of H atoms that are taken up in
%\h\ is small, and as is shown in section 7.4 of \cite{boehm1989} the
%fraction of H atoms

%Because the creation of \h\ depends not only on the normal Saha
%equation calculation of relative abundances (in our case,
%$N(H)/N(H^-)$) but also on the presence of free electrons the normal
%Saha equation can be written in the form (\citealt{collins1989})
%\beq
%\label{eq:saha}
%\frac{N(H)}{N(H^-)} = \Phi (T) P_e
%\eeq
%where $P_e$ is the electron pressure (and is hence dependent on the
%ionization states of all species present in the solar atmosphere) and
%$\Phi (T)$ is the normal partion functional form of the Saha
%ionization equation

The abundance of an ionized species can be calculated using the Saha
ionization equation, which is given by
\beq
\label{eq:saha}
\frac{n(X_{i+1})}{n(X_{i})} = \frac{(2\pi m k T)^{1.5}}{n_e h^3}\frac{2g_{i+1}}{g_{i}}e^{-\chi/kT}
\eeq
 where $\chi$ is the energy difference between the two ionization
states, $n_e$ is the number density of electrons, $k$ is the Boltzmann
constant, $h$ is the Planck constant, $T$ is the temperature, $X_i$ is
an atom $X$ in the $i$th ionization state, and $g_i$ is the degeneracy
factor corresponding to the $i$th ionization state.  Specifically
for \h\ we get
\beq
\frac{N(H)}{N(H^-)} = \frac{(2\pi m k T)^{1.5}}{n_e h^3}2\frac{1}{2}e^{-0.754\textrm{eV}/kT}.
\eeq
In practice using equation~\ref{eq:saha} is complicated by the fact that
$n_e$ is determined by electrons contributed from all species, not
just the specific atom/ion in question.  It can be calculated by
summing over the number density
of ionized particles multiplied by their degree of ionization (whether
they contribute one electron, two electrons, etc.). This is expressed by
\beq
\label{ref:ne}
n_e = \sum\limits_{i,j}j\times n_{ij}
\eeq
where $i$ represents a specific atom and $j$ represents the ionization state
of the atom, thus summing over all possible ions of all atoms.  Since,
as can be seen in equation~\ref{eq:saha} $n_{ij}$ also depends on
$n_e$ we end up with a system of equations that, when rounded out with
equation~\ref{ref:ne}, gives us the same number of equations and
unknowns.  However, values for $n_e$ in the sun have been previously
calcuated and I will use a value given in \cite{boehm1989} who
expresses the Saha equation in the logarithmic form
%\begin{align}
\begin{multline}
\log \frac{N(X_{i+1})}{N(X_i)} = \log \frac{u_{i+1}}{u_i}+\log 2 +\\
 2.5 \log T - \frac{\chi}{kT}\log e - \log P_e - 0.48
\end{multline}
%\end{align}
where log is log$_{10}$, $P_e=n_e kT$ is the
electron pressure, $u$ is the partition function
\beq
u_i = g_i + \sum\limits_{n=2}^{\infty} g_{in} e^{\chi_{in}/kT}
\eeq
which is summed over all the energy levels of a particular species.
Using the value given in \cite{boehm1989} for the electron pressure in
the solar photosphere 
($P_e = 1.5$) we can calculate the abundance of \h\ relative to H
(using 5778 K\footnote{From NASA: www.nssdc.gsfc.nasa.gov/planetary/factsheet/ sunfact.html} as $T$)
\begin{multline}
\log \frac{N(H)}{N(H^-)} = \log \frac{4}{1} + \log 2 + 9.40 - 0.658  -
1.5 - 0.48\\ = 7.7
\end{multline}
Inverting we find 
\beq
\label{eq:hminusab}
\frac{N(H^-)}{N(H)}=10^{-7.7} \approxeq 2 \times 10^{-8}.
\eeq
There is only a small fraction of H at any given time in the
photosphere that has become \h.  To understand better why \h\ in such
small abundances has such a large impact on the opacity in visible
wavelengths we must compare its abundance to other species that have
continuum opacities at visible wavelengths.  The Paschen continuum of
hydrogen (ionization from $n=3$) is the only other thing that can contribute
to continuum opacity in the visual spectral range
(\citealt{boehm1989}).  
%\cite{boehm1989} then calculates the relative
%abundances of H($n=e$) to \h\ and finds
We can calculate the relative abundance of H($n=3$) to \h.
\beq
 \frac{N(H_{n=3})}{N(H^-)} = \frac{N(H_{n=3})}{N(H_{n=1})}\frac{N(H_{n=1})}{N(H^-)}
\eeq
We can make the approximation that $N(H_{n=1})\approxeq N(H)$ since (as will be shown over the course of the next several equations) the number of H atoms with electrons in an excited state is small.
\beq
 \frac{N(H_{n=3})}{N(H^-)} = \frac{N(H_{n=3})}{N(H_{n=1})}\frac{N(H)}{N(H^-)}
\eeq
\cite{boehm1989} calculates the number of H atoms with electrons excited to $n=3$ relative to H atoms in the ground state and gets
\beq
\frac{N(H_{n=3})}{N(H_{n=1})} = 6 \times 10 ^{-10}.
\eeq
We will also plug in the value calculated by \cite{boehm1989} for $N(H)/N(H^-)$ since she used a slightly different temperature than I did in my calculation (equation~\ref{eq:hminusab}) for consistency.  She calculated $N(H)/N(H^-) \approx 3 \times 10^{-8}$.  Plugging in the values we get
\beq
\frac{N(H_{n=3})}{N(H^-)} = \frac{6 \times 10 ^{-10}}{3 \times 10^{-8}} = 2 \times 10^{-2}.
\eeq
So, there is about 100 times more \h\ than hydrogen atoms at $n=3$;
assuming that the absorption coefficients of \h\ and  hydrogen atoms
at $n=3$ are within a factor of 10 it can be clearly seen from this
calculation alone that \h\ will be the dominating contribution to the
opacity in visual wavelengths.

%The Balmer continuum is in UV
\cite{boehm1989} also shows that \h\ is only about 10 times less abundant than hydrogen atoms in $n=2$.  Since the Balmer ionization energy (corresponding to photon wavelength of 3647 \AA) occurs in the ultraviolet, where \h\ contributions to opacity are less than in the optical, and there is a relatively large fraction of $n=2$ H relative to \h\ there is a measurable increase in continuum opacity due to $n=2$ H.  %Indeed, figure~\ref{fig:bohmopacity} shows the Balmer discontinuity in the ultraviolet.

Let's put a table here and figure out how to integrate it tomorrow when I'm awake.

\begin{table}[h]
\caption{\label{tab:electroncontribution}Data on Atoms with Largest Contributions to Free Electron Count}
\begin{tabular}{c c c c}
Atom & Abundance & Ionization Potential (eV)& Relative Importance\\
H & 12.0 & 13.598 & 10$^-5$\\
Li & 1.10 & 5.392 & 0.0032\\
Na & 6.33 & 5.139 & 1.0 \\
Mg & 8.151 & 7.046 & 0.54\\
Al & 6.47 & 5.986 & 0.070\\
K  & 5.12 & 4.341 & 1.5\\
Ca & 6.36 & 6.113 & 0.58\\
Rb & 2.60 & 4.177 & 0.17\\
\end{tabular}
\end{table}


\section{The Contribution of \h\ to Stellar Opacities}
\h\ is the dominant contribution to opacity in the solar atmosphere for photons in the infrared (with wavelength $\lambda \gtrsim 1.6 \mu$m), optical, and ultraviolet regimes; since the solar luminosity peaks in this wavelength range \h\ opacity affects a significant portion of the light coming from the sun.


\section{Stellar Environments where \h\ Opacities are Dominant}
Talk about both the fact that there needs to be \h\ and also how in
some stars the Paschen, other series start to be relevant.

\section{ISM \h\, our hope to detect it}

\bibliographystyle{apj}
\bibliography{bibliography}


\end{document}
