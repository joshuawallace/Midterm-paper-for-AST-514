The 1.6$\mu$m bump is an astrophysical feature detected in extragalactic settings.  Because of the minimum of \h\ opacity around 1.6$\mu$m which causes a maximum in the observed spectral energy distributions (SEDs) of cool stars (\citealt{sawicki2002}) which feature can be detected in the integrated light from all stars in a galaxy.  For galaxies with $z\sim1$ this feature will fall into {\it Spiter}'s 3-8 $\mu$m IRAC bands.  \cite{desai2009} claim that the rest-frame 1.6$\mu$m bump can be used to select highly obscured star forming galaxies at $z\approx 2$.  %They also state that galaxies detected

\h\ ions are also important in the gas-phase formation of H$_2$ in the
interstellar medium (ISM).  The associative detachment reaction is (\citealt{draine2010})
\beq
H^- + H \rightarrow H_2 + e^- + \textrm{KE}.
\eeq
However, \h\ can also be destroyed in the ISM by interaction with
other positive ions
\beq
H^- + X^+ \rightarrow H + X
\eeq
where $X$, as before, stands for any atomic species.
