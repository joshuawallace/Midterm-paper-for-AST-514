The 1.6$\mu$m bump is an astrophysical feature detected in extragalactic settings.  Because of the minimum of \h\ opacity around 1.6$\mu$m which causes a maximum in the observed spectral energy distributions (SEDs) of cool stars (\citealt{sawicki2002}) which feature can be detected in the integrated light from all stars in a galaxy.  For galaxies with $z\trsim1$ this feature will fall into {\it Spiter}'s 3-8 $\mu$m IRAC bands.  \cite{desai2009} claim that the rest-frame 1.6$\mu$m bump can be used to select highly obscured starforming galaxies at $z\approx 2$.  %They also state that galaxies detected

caused by the minimum in the opacity of \h\ can be a feature used to determine photometric redshifts of galaxies at $z\gtrsim1.5$
