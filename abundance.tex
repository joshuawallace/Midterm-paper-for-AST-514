The \h\ abundance depends on several factors: the abundance of H, the
abundance of free electrons (ionized from other species present in the
atmosphere, including H), and the factors that lead to the conversion
of \h\ to H: photoionization and collisions with other particles.  All
of these factors have some sort of dependence on temperature.

%The solar atmosphere is made up about 70\% by mass of H.  As will be
%shown in what follows, the fraction of H atoms that are taken up in
%\h\ is small, and as is shown in section 7.4 of \cite{boehm1989} the
%fraction of H atoms

Because the creation of \h\ depends not only on the normal Saha
equation calculation of relative abundances (in our case,
$N(H)/N(H^-)$) but also on the presence of free electrons the normal
Saha equation can be written in the form (\citealt{collins1989})
\beq
\frac{N(H)}{N(H^-)} = \Phi (T) P_e
\eeq
where $P_e$ is the electron pressure (and is hence dependent on the
ionization states of all species present in the solar atmosphere) and
$\Phi (T)$ is the normal partion functional form of the Saha
ionization equation
\beq
\Phi (T) = \frac{(2\pi m k T)^{1.5}}{n_e h^3}\frac{2g_{H}}{g_{H^-}}e^{-\chi/kT}
\eeq
where $\chi$ is the energy difference between the two ionization
states and $g_H$ is the degeneracy of states for H and $g_{H^-}$ is
the corresponding value for \h, and $n_e$ is the number density of electrons.
