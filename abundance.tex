The \h\ abundance depends on several factors: the abundance of H, the
abundance of free electrons (ionized from other species present in the
atmosphere, including H), and the factors that lead to the conversion
of \h\ to H: photoionization and collisions with other particles.  All
of these factors have some sort of dependence on temperature.

%The solar atmosphere is made up about 70\% by mass of H.  As will be
%shown in what follows, the fraction of H atoms that are taken up in
%\h\ is small, and as is shown in section 7.4 of \cite{boehm1989} the
%fraction of H atoms

%Because the creation of \h\ depends not only on the normal Saha
%equation calculation of relative abundances (in our case,
%$N(H)/N(H^-)$) but also on the presence of free electrons the normal
%Saha equation can be written in the form (\citealt{collins1989})
%\beq
%\label{eq:saha}
%\frac{N(H)}{N(H^-)} = \Phi (T) P_e
%\eeq
%where $P_e$ is the electron pressure (and is hence dependent on the
%ionization states of all species present in the solar atmosphere) and
%$\Phi (T)$ is the normal partion functional form of the Saha
%ionization equation

The abundance of an ionized species can be calculated using the Saha
ionization equation, which is given by
\beq
\label{eq:saha}
\frac{n(X_{i+1})}{n(X_{i})} = \frac{(2\pi m k T)^{1.5}}{n_e h^3}\frac{2g_{i+1}}{g_{i}}e^{-\chi/kT}
\eeq
 where $\chi$ is the energy difference between the two ionization
states, $n_e$ is the number density of electrons, $k$ is the Boltzmann
constant, $h$ is the Planck constant, $T$ is the temperature, $X_i$ is
an atom $X$ in the $i$th ionization state, and $g_i$ is the degeneracy
factor corresponding to the $i$th ionization state.  Specifically
for \h\ we get
\beq
\frac{N(H)}{N(H^-)} = \frac{(2\pi m k T)^{1.5}}{n_e h^3}2\frac{1}{2}e^{-0.754\textrm{eV}/kT}.
\eeq
In practice using equation~\ref{eq:saha} is complicated by the fact that
$n_e$ is determined by electrons contributed from all species, not
just the specific atom/ion in question.  It can be calculated by
summing over the number density
of ionized particles multiplied by their degree of ionization (whether
they contribute one electron, two electrons, etc.). This is expressed by
\beq
\label{ref:ne}
n_e = \sum\limits_{i,j}j\times n_{ij}
\eeq
where $i$ represents a specific atom and $j$ represents the ionization state
of the atom, thus summing over all possible ions of all atoms.  Since,
as can be seen in equation~\ref{eq:saha} $n_{ij}$ also depends on
$n_e$ we end up with a system of equations that, when rounded out with
equation~\ref{ref:ne}, gives us the same number of equations and
unknowns.  However, values for $n_e$ in the sun have been previously
calcuated and I will use a value given in \cite{boehm1989} who
expresses the Saha equation in the logarithmic form
%\begin{align}
\begin{multline}
\log \frac{N(X_{i+1})}{N(X_i)} = \log \frac{u_{i+1}}{u_i}+\log 2 +\\
 2.5 \log T - \frac{\chi}{kT}\log e - \log P_e - 0.48
\end{multline}
%\end{align}
where log is log$_{10}$, $P_e=n_e kT$ is the
electron pressure, $u$ is the partition function
\beq
u_i = g_i + \sum\limits_{n=2}^{\infty} g_{in} e^{\chi_{in}/kT}
\eeq
which is summed over all the energy levels of a particular species.
Using the value given in \cite{boehm1989} for the electron pressure in
the solar photosphere 
($P_e = 1.5$) we can calculate the abundance of \h\ relative to H
(using 5778 K\footnote{From NASA: www.nssdc.gsfc.nasa.gov/planetary/factsheet/sunfact.html} as $T$)
\begin{multline}
\log \frac{N(H)}{N(H^-)} = \log \frac{4}{1} + \log 2 + 9.40 - 0.658  -
1.5 - 0.48\\ = 7.7
\end{multline}
Inverting we find 
\beq
\frac{N(H^-)}{N(H)}=10^{-7.7} \approxeq 2 \times 10^{-8}.
\eeq
There is only a small fraction of H at any given time in the
photosphere that has become \h.  To understand better why \h\ in such
small abundances has such a large impact on the opacity in visible
wavelengths we must compare its abundance to other species that have
continuum opacities at visible wavelengths.  The Paschen continuum of
hydrogen (ionization from $n=3$) is the only other thing that can contribute
to continuum opacity in the visual spectral range
(\citealt{boehm1989}).  \cite{boehm1989} then calculates the relative
abundances of H($n=e$) to \h\ and finds
\beq
\frac{N_H(n=3)}{N(H^-)} = 2 \times 10^{-2}.
\eeq
So, there is about 100 times more \h\ than hydrogen atoms at $n=3$;
assuming that the absorption coefficients of \h\ and  hydrogen atoms
at $n=3$ are within a factor of 10 it can be clearly seen from this
calculation alone that \h\ will be the dominating contribution to the
opacity in visual wavelengths.

%The Balmer continuum is in UV
