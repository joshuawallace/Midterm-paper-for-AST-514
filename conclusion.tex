\h\ provides the dominant contribution to solar photospheric opacity
in the near-infrared and visible regimes, as well as a significant
contribution in the near-ultraviolet.  At wavelengths less $\sim
1.6 \mu$m contributions from bound-free absorption on \h\ dominates
the opacity over free-free, which dominates at wavelengths less than
the \h\ ionization cutoff.  The examination and calculation of these
opacities have had a long history and are well quantified and
understood.  From what I understand there is little-to-no work to be
done in a solar context in regard to \h\ opacities.  %There is,
%however, active research with \h\ in regards to high energy physics
Additionally, the dominance of \h\ opacity in the sun does not carry
over to other types of stars; in stars hotter than our sun opacity
contributions H and He become increasingly important with temperature
as the energy of the spectral emission from the star increases; in K
stars molecules start to dominate the opacity
(\citealt{collins1989}). However, in late-F to G stars \h\ opacity is
important. 
